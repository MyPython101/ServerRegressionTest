% Options for packages loaded elsewhere
\PassOptionsToPackage{unicode}{hyperref}
\PassOptionsToPackage{hyphens}{url}
%
\documentclass[
  ignorenonframetext,
]{beamer}
\title{Simple Test for PHP (Unit Testing)}
\author{Truc Huynh \& Mohammed Alswairki}
\date{3/5/2022}

\usepackage{pgfpages}
\setbeamertemplate{caption}[numbered]
\setbeamertemplate{caption label separator}{: }
\setbeamercolor{caption name}{fg=normal text.fg}
\beamertemplatenavigationsymbolsempty
% Prevent slide breaks in the middle of a paragraph
\widowpenalties 1 10000
\raggedbottom
\setbeamertemplate{part page}{
  \centering
  \begin{beamercolorbox}[sep=16pt,center]{part title}
    \usebeamerfont{part title}\insertpart\par
  \end{beamercolorbox}
}
\setbeamertemplate{section page}{
  \centering
  \begin{beamercolorbox}[sep=12pt,center]{part title}
    \usebeamerfont{section title}\insertsection\par
  \end{beamercolorbox}
}
\setbeamertemplate{subsection page}{
  \centering
  \begin{beamercolorbox}[sep=8pt,center]{part title}
    \usebeamerfont{subsection title}\insertsubsection\par
  \end{beamercolorbox}
}
\AtBeginPart{
  \frame{\partpage}
}
\AtBeginSection{
  \ifbibliography
  \else
    \frame{\sectionpage}
  \fi
}
\AtBeginSubsection{
  \frame{\subsectionpage}
}
\usepackage{amsmath,amssymb}
\usepackage{lmodern}
\usepackage{iftex}
\ifPDFTeX
  \usepackage[T1]{fontenc}
  \usepackage[utf8]{inputenc}
  \usepackage{textcomp} % provide euro and other symbols
\else % if luatex or xetex
  \usepackage{unicode-math}
  \defaultfontfeatures{Scale=MatchLowercase}
  \defaultfontfeatures[\rmfamily]{Ligatures=TeX,Scale=1}
\fi
% Use upquote if available, for straight quotes in verbatim environments
\IfFileExists{upquote.sty}{\usepackage{upquote}}{}
\IfFileExists{microtype.sty}{% use microtype if available
  \usepackage[]{microtype}
  \UseMicrotypeSet[protrusion]{basicmath} % disable protrusion for tt fonts
}{}
\makeatletter
\@ifundefined{KOMAClassName}{% if non-KOMA class
  \IfFileExists{parskip.sty}{%
    \usepackage{parskip}
  }{% else
    \setlength{\parindent}{0pt}
    \setlength{\parskip}{6pt plus 2pt minus 1pt}}
}{% if KOMA class
  \KOMAoptions{parskip=half}}
\makeatother
\usepackage{xcolor}
\IfFileExists{xurl.sty}{\usepackage{xurl}}{} % add URL line breaks if available
\IfFileExists{bookmark.sty}{\usepackage{bookmark}}{\usepackage{hyperref}}
\hypersetup{
  pdftitle={Simple Test for PHP (Unit Testing)},
  pdfauthor={Truc Huynh \& Mohammed Alswairki},
  hidelinks,
  pdfcreator={LaTeX via pandoc}}
\urlstyle{same} % disable monospaced font for URLs
\newif\ifbibliography
\usepackage{color}
\usepackage{fancyvrb}
\newcommand{\VerbBar}{|}
\newcommand{\VERB}{\Verb[commandchars=\\\{\}]}
\DefineVerbatimEnvironment{Highlighting}{Verbatim}{commandchars=\\\{\}}
% Add ',fontsize=\small' for more characters per line
\usepackage{framed}
\definecolor{shadecolor}{RGB}{248,248,248}
\newenvironment{Shaded}{\begin{snugshade}}{\end{snugshade}}
\newcommand{\AlertTok}[1]{\textcolor[rgb]{0.94,0.16,0.16}{#1}}
\newcommand{\AnnotationTok}[1]{\textcolor[rgb]{0.56,0.35,0.01}{\textbf{\textit{#1}}}}
\newcommand{\AttributeTok}[1]{\textcolor[rgb]{0.77,0.63,0.00}{#1}}
\newcommand{\BaseNTok}[1]{\textcolor[rgb]{0.00,0.00,0.81}{#1}}
\newcommand{\BuiltInTok}[1]{#1}
\newcommand{\CharTok}[1]{\textcolor[rgb]{0.31,0.60,0.02}{#1}}
\newcommand{\CommentTok}[1]{\textcolor[rgb]{0.56,0.35,0.01}{\textit{#1}}}
\newcommand{\CommentVarTok}[1]{\textcolor[rgb]{0.56,0.35,0.01}{\textbf{\textit{#1}}}}
\newcommand{\ConstantTok}[1]{\textcolor[rgb]{0.00,0.00,0.00}{#1}}
\newcommand{\ControlFlowTok}[1]{\textcolor[rgb]{0.13,0.29,0.53}{\textbf{#1}}}
\newcommand{\DataTypeTok}[1]{\textcolor[rgb]{0.13,0.29,0.53}{#1}}
\newcommand{\DecValTok}[1]{\textcolor[rgb]{0.00,0.00,0.81}{#1}}
\newcommand{\DocumentationTok}[1]{\textcolor[rgb]{0.56,0.35,0.01}{\textbf{\textit{#1}}}}
\newcommand{\ErrorTok}[1]{\textcolor[rgb]{0.64,0.00,0.00}{\textbf{#1}}}
\newcommand{\ExtensionTok}[1]{#1}
\newcommand{\FloatTok}[1]{\textcolor[rgb]{0.00,0.00,0.81}{#1}}
\newcommand{\FunctionTok}[1]{\textcolor[rgb]{0.00,0.00,0.00}{#1}}
\newcommand{\ImportTok}[1]{#1}
\newcommand{\InformationTok}[1]{\textcolor[rgb]{0.56,0.35,0.01}{\textbf{\textit{#1}}}}
\newcommand{\KeywordTok}[1]{\textcolor[rgb]{0.13,0.29,0.53}{\textbf{#1}}}
\newcommand{\NormalTok}[1]{#1}
\newcommand{\OperatorTok}[1]{\textcolor[rgb]{0.81,0.36,0.00}{\textbf{#1}}}
\newcommand{\OtherTok}[1]{\textcolor[rgb]{0.56,0.35,0.01}{#1}}
\newcommand{\PreprocessorTok}[1]{\textcolor[rgb]{0.56,0.35,0.01}{\textit{#1}}}
\newcommand{\RegionMarkerTok}[1]{#1}
\newcommand{\SpecialCharTok}[1]{\textcolor[rgb]{0.00,0.00,0.00}{#1}}
\newcommand{\SpecialStringTok}[1]{\textcolor[rgb]{0.31,0.60,0.02}{#1}}
\newcommand{\StringTok}[1]{\textcolor[rgb]{0.31,0.60,0.02}{#1}}
\newcommand{\VariableTok}[1]{\textcolor[rgb]{0.00,0.00,0.00}{#1}}
\newcommand{\VerbatimStringTok}[1]{\textcolor[rgb]{0.31,0.60,0.02}{#1}}
\newcommand{\WarningTok}[1]{\textcolor[rgb]{0.56,0.35,0.01}{\textbf{\textit{#1}}}}
\setlength{\emergencystretch}{3em} % prevent overfull lines
\providecommand{\tightlist}{%
  \setlength{\itemsep}{0pt}\setlength{\parskip}{0pt}}
\setcounter{secnumdepth}{-\maxdimen} % remove section numbering
\ifLuaTeX
  \usepackage{selnolig}  % disable illegal ligatures
\fi

\begin{document}
\frame{\titlepage}

\begin{frame}{Keyword (*)}
\protect\hypertarget{keyword}{}
\begin{itemize}
\tightlist
\item
  JUnit: is a regression unit testing framework for Java programming
  languages
\item
  unit test: is a testing method where the smallest testable parts of a
  software are tested
\item
  xUnit: xUnit.net is a free, open source, community-focused unit
  testing tool for the .NET Framework (1)
\item
  PHP: a server side scripting languages
\item
  Eclipse IDE: Integrated development environment for Java development
\item
  test case: is a document, which has a set of test data, preconditions,
  expected results and post-conditions, developed for a particular test
  scenario in order to verify compliance against a specific requirement
  (1)
\item
  Test suite: grouping test cases into a test script that can run every
  test for the application (1)
\item
  Test Driven Development (TDD): test cases for each functionality are
  created and tested first and if the test fails then the new code is
  written in order to pass the test and making code simple and bug-free
  (1)
\end{itemize}
\end{frame}

\begin{frame}{Introduction to SimpleTest}
\protect\hypertarget{introduction-to-simpletest}{}
\begin{itemize}
\tightlist
\item
  SimpleTest is a regression testing framework that built around test
  cases
\item
  Test PHP only (not Python, not anything else)
\item
  Completed PHP developer test solution ( unit test, automation test,
  GUI test, mock object, TDD (*)\ldots)
\end{itemize}
\end{frame}

\begin{frame}{SimpleTest Pre-requirement}
\protect\hypertarget{simpletest-pre-requirement}{}
\begin{itemize}
\tightlist
\item
  Understand of testing methods(in this case Unit Testing)
\item
  Understand of PHP web development language
\item
  Know how to program (design for developer and tester)
\end{itemize}
\end{frame}

\begin{frame}{SimpleTest Features}
\protect\hypertarget{simpletest-features}{}
\begin{itemize}
\tightlist
\item
  Easy to use and extend
\item
  Automation testing (schedule scripts)
\item
  Completed PHP developer test solution
\item
  xUnit (*) style test cases
\item
  Support mock objects(*)
\item
  Built in web browser (no need for selenium)
\item
  Can Navigate websites, fill in form, authentication, web tester
\item
  \emph{\textbf{Retrieved from (1)}}
\end{itemize}
\end{frame}

\begin{frame}{Download Simple Test for PHP}
\protect\hypertarget{download-simple-test-for-php}{}
\begin{itemize}
\tightlist
\item
  Download from SourceForge.net
\item
  Download as a plugin for
  \href{https://sourceforge.net/projects/simpletest/files/eclipse\%20plugin/}{Eclipse
  IDE (*)} also from SourceForge.net
\item
  \emph{\textbf{Retrieved from (1)}}
\end{itemize}
\end{frame}

\begin{frame}{How to Use Simple Test for PHP}
\protect\hypertarget{how-to-use-simple-test-for-php}{}
\begin{itemize}
\tightlist
\item
  Use as PHP script, develop along with the source code
\item
  Use Simple Test to develop test case, test suite (*)
\item
  \emph{\textbf{Retrieved from (1)}}
\end{itemize}
\end{frame}

\begin{frame}{What is Unit Testing}
\protect\hypertarget{what-is-unit-testing}{}
\begin{itemize}
\tightlist
\item
  Unit Testing: Not only test if a unit (collection of statements:
  statements, function, block, object) is working but what happen when
  it is not working.
\item
  Usually perform by developer
\end{itemize}
\end{frame}

\begin{frame}{What is PHP}
\protect\hypertarget{what-is-php}{}
\begin{itemize}
\tightlist
\item
  PHP is an acronym for ``PHP: Hypertext Pre-processor''
\item
  PHP is a widely-used, open source scripting language
\item
  PHP scripts are executed on the server
\item
  PHP is free to download and use
\item
  \emph{\textbf{Retrieved from (5)}}
\end{itemize}
\end{frame}

\begin{frame}{What is PHP files}
\protect\hypertarget{what-is-php-files}{}
\begin{itemize}
\tightlist
\item
  PHP files can contain text, HTML, CSS, JavaScript, and PHP code
\item
  PHP code is executed on the server, and the result is returned to the
  browser as plain HTML
\item
  PHP files have extension ``.php''
\item
  \emph{\textbf{Retrieved from (5)}}
\end{itemize}
\end{frame}

\begin{frame}{SimpleTest classes}
\protect\hypertarget{simpletest-classes}{}
\begin{itemize}
\tightlist
\item
  use \emph{\textbf{extend}} keyword to implement any class in
  SimpleTest framework
\item
  Most use package:

  \begin{itemize}
  \tightlist
  \item
    `simpletest/autorun.php': include for all kind of simple test
  \item
    `simpletest/web\_tester.php': web test (GUI testing)
  \item
    `simpletest/mock\_objects.php': generate mock object
  \end{itemize}
\item
  Most use class/interface:

  \begin{itemize}
  \tightlist
  \item
    extends TestSuite: test suites
  \item
    extends UnitTestCase: unit test case
  \item
    Mock : generate mock object
  \end{itemize}
\end{itemize}
\end{frame}

\begin{frame}{Unit Test cases}
\protect\hypertarget{unit-test-cases}{}
\begin{itemize}
\tightlist
\item
  The simplest type of all test case is the unit test
\item
  implement SimpleTest framework by extend the UnitTestCase.
\item
  This class of test case includes standard tests for equality,
  references and pattern matching
\end{itemize}
\end{frame}

\begin{frame}[fragile]{UnitTestCase Class}
\protect\hypertarget{unittestcase-class}{}
Some function of UnitTestCase class, the default for SimpleTest

\begin{Shaded}
\begin{Highlighting}[]
\NormalTok{assertTrue(}\VariableTok{$x}\NormalTok{)                  }\CommentTok{// Fail if $x is false}
\NormalTok{assertFalse(}\VariableTok{$x}\NormalTok{)                 }\CommentTok{// Fail if $x is true}
\NormalTok{assertNull(}\VariableTok{$x}\NormalTok{)                  }\CommentTok{// Fail if $x is set}
\NormalTok{assertNotNull(}\VariableTok{$x}\NormalTok{)               }\CommentTok{// Fail if $x not set}
\NormalTok{assertIsA(}\VariableTok{$x}\OtherTok{,} \VariableTok{$t}\NormalTok{)               }\CommentTok{// Fail if $x is not the class or type $t}
\NormalTok{assertNotA(}\VariableTok{$x}\OtherTok{,} \VariableTok{$t}\NormalTok{)              }\CommentTok{// Fail if $x is of the class or type $t}
\NormalTok{assertEqual(}\VariableTok{$x}\OtherTok{,} \VariableTok{$y}\NormalTok{)             }\CommentTok{// Fail if $x == $y is false}
\NormalTok{assertNotEqual(}\VariableTok{$x}\OtherTok{,} \VariableTok{$y}\NormalTok{)          }\CommentTok{// Fail if $x == $y is true}
\NormalTok{assertWithinMargin(}\VariableTok{$x}\OtherTok{,} \VariableTok{$y}\OtherTok{,} \VariableTok{$m}\NormalTok{)  }\CommentTok{// Fail if abs($x {-} $y) \textless{} $m is false}
\NormalTok{assertOutsideMargin(}\VariableTok{$x}\OtherTok{,} \VariableTok{$y}\OtherTok{,} \VariableTok{$m}\NormalTok{) }\CommentTok{// Fail if abs($x {-} $y) \textless{} $m is true}
\CommentTok{// see more at http://simpletest.sourceforge.net/en/unit\_test\_documentation.html}
\end{Highlighting}
\end{Shaded}
\end{frame}

\begin{frame}[fragile]{Convenience Methods (UnitTestCase Class)}
\protect\hypertarget{convenience-methods-unittestcase-class}{}
\begin{itemize}
\tightlist
\item
  The test cases also have some convenience methods for debugging code
  or extending the suite\ldots{}
\end{itemize}

\begin{Shaded}
\begin{Highlighting}[]
\NormalTok{setUp()                 }\CommentTok{// Runs this before each test method}
\NormalTok{tearDown()              }\CommentTok{// Runs this after each test method}
\NormalTok{pass()                  }\CommentTok{// Sends a test pass}
\NormalTok{fail()                  }\CommentTok{// Sends a test failure}
\BuiltInTok{error}\NormalTok{()                 }\CommentTok{// Sends an exception event}
\NormalTok{signal(}\VariableTok{$type}\OtherTok{,} \VariableTok{$payload}\NormalTok{) }\CommentTok{// Sends a user defined message to the test reporter}
\NormalTok{dump(}\VariableTok{$var}\NormalTok{)              }\CommentTok{// Does a formatted print\_r() for quick and dirty debugging}
\end{Highlighting}
\end{Shaded}
\end{frame}

\begin{frame}[fragile]{Simple Use Case (Unit Test)}
\protect\hypertarget{simple-use-case-unit-test}{}
\begin{itemize}
\tightlist
\item
  Simple test case that check a file has been created by the Writer
  object
\end{itemize}

\begin{Shaded}
\begin{Highlighting}[]
\KeywordTok{\textless{}?php}
\CommentTok{// The "autorun.php" file does more than just include the SimpleTest files, it also runs our test for us}
\KeywordTok{require\_once}\NormalTok{(}\StringTok{\textquotesingle{}simpletest/autorun.php\textquotesingle{}}\NormalTok{)}\OtherTok{;}
\KeywordTok{require\_once}\NormalTok{(}\StringTok{\textquotesingle{}../classes/writer.php\textquotesingle{}}\NormalTok{)}\OtherTok{;}
\CommentTok{// use extends UnitTestCase to implement simple test}
\KeywordTok{class} \ConstantTok{F}\NormalTok{ileTestCase }\KeywordTok{extends} \ConstantTok{U}\NormalTok{nitTestCase \{}
    \KeywordTok{function} \ConstantTok{F}\NormalTok{ileTestCase() \{}
        \VariableTok{$this}\NormalTok{{-}\textgreater{}}\ConstantTok{U}\NormalTok{nitTestCase(}\StringTok{\textquotesingle{}File test\textquotesingle{}}\NormalTok{)}\OtherTok{;}\NormalTok{\}    }\CommentTok{// Constructor}
    \CommentTok{//run just before each and every test method}
    \KeywordTok{function}\NormalTok{ setUp() \{}
        \PreprocessorTok{@}\FunctionTok{unlink}\NormalTok{(}\StringTok{\textquotesingle{}../temp/test.txt\textquotesingle{}}\NormalTok{)}\OtherTok{;}\NormalTok{\}}
    \CommentTok{//run just before each and every test method}
    \KeywordTok{function}\NormalTok{ tearDown() \{}
        \PreprocessorTok{@}\FunctionTok{unlink}\NormalTok{(}\StringTok{\textquotesingle{}../temp/test.txt\textquotesingle{}}\NormalTok{)}\OtherTok{;}\NormalTok{\}}
    \CommentTok{// When a test case runs, it will search for any method that starts with the string "test" and execute that method.}
    \KeywordTok{function}\NormalTok{ testCreation() \{}
        \VariableTok{$writer} \OperatorTok{=} \OperatorTok{\&}\KeywordTok{new} \ConstantTok{F}\NormalTok{ileWriter(}\StringTok{\textquotesingle{}../temp/test.txt\textquotesingle{}}\NormalTok{)}\OtherTok{;}
        \VariableTok{$writer}\NormalTok{{-}\textgreater{}write(}\StringTok{\textquotesingle{}Hello\textquotesingle{}}\NormalTok{)}\OtherTok{;}
        \CommentTok{// using assertTrue function validate if the test.txt is exit, if true print \textquotesingle{}File created\textquotesingle{}}
        \VariableTok{$this}\NormalTok{{-}\textgreater{}assertTrue(}\FunctionTok{file\_exists}\NormalTok{(}\StringTok{\textquotesingle{}../temp/test.txt\textquotesingle{}}\NormalTok{)}\OtherTok{,} \StringTok{\textquotesingle{}File created\textquotesingle{}}\NormalTok{)}\OtherTok{;}
\NormalTok{\}\}}\KeywordTok{?\textgreater{}}
\end{Highlighting}
\end{Shaded}
\end{frame}

\begin{frame}[fragile]{Group Test}
\protect\hypertarget{group-test}{}
\begin{itemize}
\tightlist
\item
  Group test cases to create a test suites
\item
  Template of test suite, `log\_test.php' content
\end{itemize}

\begin{Shaded}
\begin{Highlighting}[]

\KeywordTok{\textless{}?php}
\CommentTok{// when list as abstract MyFileTestCase will not execute}
\KeywordTok{abstract} \KeywordTok{class} \ConstantTok{M}\NormalTok{yFileTestCase }\KeywordTok{extends} \ConstantTok{U}\NormalTok{nitTestCase \{ }
 \CommentTok{//Do sometest}
\NormalTok{ \}}

\KeywordTok{class} \ConstantTok{F}\NormalTok{ileTester }\KeywordTok{extends} \ConstantTok{M}\NormalTok{yFileTestCase \{ }
 \CommentTok{//Do sometest}
\NormalTok{ \}}

\KeywordTok{class} \ConstantTok{S}\NormalTok{ocketTester }\KeywordTok{extends} \ConstantTok{U}\NormalTok{nitTestCase \{ }
 \CommentTok{//Do sometest}
\NormalTok{ \}}
\KeywordTok{?\textgreater{}}
\end{Highlighting}
\end{Shaded}
\end{frame}

\begin{frame}[fragile]{Group Test (cont.)}
\protect\hypertarget{group-test-cont.}{}
\begin{itemize}
\tightlist
\item
  Script 2: Test Suite `all-test.php'
\end{itemize}

\begin{Shaded}
\begin{Highlighting}[]
\KeywordTok{\textless{}?php}
\KeywordTok{require\_once}\NormalTok{(}\StringTok{\textquotesingle{}simpletest/autorun.php\textquotesingle{}}\NormalTok{)}\OtherTok{;}

\CommentTok{// Note that allTest extends TestSuite}
\KeywordTok{class} \ConstantTok{A}\NormalTok{llTests }\KeywordTok{extends} \ConstantTok{T}\NormalTok{estSuite \{}
    \KeywordTok{function} \ConstantTok{A}\NormalTok{llTests() \{}
        \VariableTok{$this}\NormalTok{{-}\textgreater{}}\ConstantTok{T}\NormalTok{estSuite(}\StringTok{\textquotesingle{}All tests\textquotesingle{}}\NormalTok{)}\OtherTok{;}
        \VariableTok{$this}\NormalTok{{-}\textgreater{}addFile(}\StringTok{\textquotesingle{}log\_test.php\textquotesingle{}}\NormalTok{)}\OtherTok{;}
\NormalTok{    \}}
\NormalTok{\}}
\KeywordTok{?\textgreater{}}
\end{Highlighting}
\end{Shaded}
\end{frame}

\begin{frame}{What is Mock Objects}
\protect\hypertarget{what-is-mock-objects}{}
\begin{itemize}
\tightlist
\item
  Technique for improving the design of code within Test-Driven
  Development
\item
  Simulated Object that simulate the behavior of a testable unit
\item
  Mock objects help isolate the component being tested from the
  components it depends on (2)
\end{itemize}
\end{frame}

\begin{frame}{Use case 1 (Mock Objects)}
\protect\hypertarget{use-case-1-mock-objects}{}
\begin{itemize}
\tightlist
\item
  Pre condition:

  \begin{itemize}
  \tightlist
  \item
    We test a web app
  \item
    Our app make HTTP requests to an external services
  \item
    External services always perform as expected but what if the
    services fail?
  \item
    Temporary change in the behavior of these external services can
    cause immediate failures within our test suite.
  \end{itemize}
\item
  Benefit of using Mock Objects:

  \begin{itemize}
  \tightlist
  \item
    We test our app in a controlled environment
  \item
    Replacing actual request with a mock object (simulated object)
  \item
    Successfully simulate external services outage and design a response
    (in a predicted way)
  \end{itemize}
\end{itemize}
\end{frame}

\begin{frame}[fragile]{Use Case 2 (Mock Objects)}
\protect\hypertarget{use-case-2-mock-objects}{}
\begin{itemize}
\tightlist
\item
  Technology

  \begin{itemize}
  \tightlist
  \item
    PHP
  \item
    SimpleTest
  \end{itemize}
\item
  Scenario:

  \begin{itemize}
  \tightlist
  \item
    Test database connection, simulate database being down without
    creating new real broken database(s)
  \end{itemize}
\item
  Step

  \begin{itemize}
  \tightlist
  \item
    Create DatabaseConnection class store in script
    `database\_connection.php':
  \end{itemize}
\end{itemize}

\begin{Shaded}
\begin{Highlighting}[]
\KeywordTok{\textless{}?php}
\CommentTok{// database\_connection.php}
\KeywordTok{class} \ConstantTok{D}\NormalTok{atabaseConnection \{}
    \KeywordTok{function} \ConstantTok{D}\NormalTok{atabaseConnection() \{ }\CommentTok{// do some\}}
    
    \KeywordTok{function}\NormalTok{ query() \{ }\CommentTok{// do some query\}}
    
    \KeywordTok{function}\NormalTok{ selectQuery() \{ }\CommentTok{// do some searches\}}
\NormalTok{\}}
\KeywordTok{?\textgreater{}}
\end{Highlighting}
\end{Shaded}
\end{frame}

\begin{frame}[fragile]{Use Case 2 (Mock Objects) (cont.)}
\protect\hypertarget{use-case-2-mock-objects-cont.}{}
\begin{itemize}
\tightlist
\item
  run the generator to create mock version of DatabaseConnection
\end{itemize}

\begin{Shaded}
\begin{Highlighting}[]
\KeywordTok{\textless{}?php}
\KeywordTok{require\_once}\NormalTok{(}\StringTok{\textquotesingle{}simpletest/unit\_tester.php\textquotesingle{}}\NormalTok{)}\OtherTok{;}
\KeywordTok{require\_once}\NormalTok{(}\StringTok{\textquotesingle{}simpletest/mock\_objects.php\textquotesingle{}}\NormalTok{)}\OtherTok{;}    \CommentTok{//include the mock object library}
\KeywordTok{require\_once}\NormalTok{(}\StringTok{\textquotesingle{}database\_connection.php\textquotesingle{}}\NormalTok{)}\OtherTok{;}

\ConstantTok{M}\NormalTok{ock::generate(}\StringTok{\textquotesingle{}DatabaseConnection\textquotesingle{}}\NormalTok{)}\OtherTok{;}           \CommentTok{// run the generator to create a mock version of DatabaseConnection}
\KeywordTok{?\textgreater{}}
\end{Highlighting}
\end{Shaded}

\begin{itemize}
\tightlist
\item
  This generates a clone class called MockDatabaseConnection.
\item
  We can now create instances of the new class within our test case
\end{itemize}

\begin{Shaded}
\begin{Highlighting}[]
\KeywordTok{\textless{}?php}
\KeywordTok{class} \ConstantTok{M}\NormalTok{yTestCase }\KeywordTok{extends} \ConstantTok{U}\NormalTok{nitTestCase \{}
    \KeywordTok{function}\NormalTok{ testSomething() \{}
        \VariableTok{$connection} \OperatorTok{=} \OperatorTok{\&}\KeywordTok{new} \ConstantTok{M}\NormalTok{ockDatabaseConnection()}\OtherTok{;}
\NormalTok{    \}}
\NormalTok{\}}
\KeywordTok{?\textgreater{}}
\end{Highlighting}
\end{Shaded}
\end{frame}

\begin{frame}{Python Unit Test}
\protect\hypertarget{python-unit-test}{}
\begin{itemize}
\tightlist
\item
  The unittest unit testing framework was originally inspired by JUnit
  (*) and has a similar flavor as major unit testing frameworks in other
  languages
\item
  It supports test automation
\item
  Sharing of setup and shutdown code for tests
\item
  Aggregation of tests into collections, and independence of the tests
  from the reporting framework(4)
\end{itemize}
\end{frame}

\begin{frame}[fragile]{Python Mock Object}
\protect\hypertarget{python-mock-object}{}
\begin{itemize}
\tightlist
\item
  Python with json
\end{itemize}

\begin{Shaded}
\begin{Highlighting}[]
\ImportTok{import}\NormalTok{ json}

\CommentTok{\# a Python object (dict):}
\NormalTok{x }\OperatorTok{=}\NormalTok{ \{  }\StringTok{"name"}\NormalTok{: }\StringTok{"John"}\NormalTok{,}\StringTok{"age"}\NormalTok{: }\DecValTok{30}\NormalTok{,}\StringTok{"city"}\NormalTok{: }\StringTok{"New York"}\NormalTok{\}}

\CommentTok{\# convert into JSON:}
\NormalTok{y }\OperatorTok{=}\NormalTok{ json.dumps(x)}

\CommentTok{\# the result is a JSON string:}
\BuiltInTok{print}\NormalTok{(y)}
\end{Highlighting}
\end{Shaded}
\end{frame}

\begin{frame}[fragile]{Python Mock Object (cont.)}
\protect\hypertarget{python-mock-object-cont.}{}
\begin{itemize}
\tightlist
\item
  Using Python Mock Object of unittest package
\end{itemize}

\begin{Shaded}
\begin{Highlighting}[]
\OperatorTok{\textgreater{}\textgreater{}\textgreater{}}\NormalTok{ from }\ExtensionTok{unittest.mock}\NormalTok{ import Mock}
\OperatorTok{\textgreater{}\textgreater{}\textgreater{}}\NormalTok{ import }\ExtensionTok{json}
\OperatorTok{\textgreater{}\textgreater{}\textgreater{}}\NormalTok{ json }\ExtensionTok{=}\NormalTok{ Mock}\ErrorTok{(}\KeywordTok{)}
\OperatorTok{\textgreater{}\textgreater{}\textgreater{}}\NormalTok{ json.dumps}\KeywordTok{()}
\OperatorTok{\textless{}}\NormalTok{Mock }\VariableTok{name}\OperatorTok{=}\StringTok{\textquotesingle{}mock.dumps()\textquotesingle{}} \VariableTok{id}\OperatorTok{=}\StringTok{\textquotesingle{}4392249776\textquotesingle{}}\OperatorTok{\textgreater{}}
\end{Highlighting}
\end{Shaded}

\begin{itemize}
\tightlist
\item
  Unlike the real dumps(), this mocked method requires no arguments. In
  fact, it will accept any arguments that you pass to it.
\item
  The return value of dumps() is also a Mock object. The capability of
  Mock to recursively define other mocks allows for you to use mocks in
  complex situations.
\end{itemize}
\end{frame}

\begin{frame}{Reference}
\protect\hypertarget{reference}{}
\begin{itemize}
\tightlist
\item
  \href{http://simpletest.sourceforge.net/en/start-testing.html}{Simple
  Test for PHP} (1)
\item
  \href{https://searchsoftwarequality.techtarget.com/definition/mock-object}{mock
  object} (2)
\item
  \href{https://realpython.com/python-mock-library/}{Understanding the
  Python Mock Object Library} (3)
\item
  \href{https://docs.python.org/3/library/unittest.html}{Python Unit
  test} (4)
\item
  \href{https://www.w3schools.com/PHP/php_intro.asp\#:~:text=What\%20is\%20PHP\%3F\%201\%20PHP\%20is\%20an\%20acronym,4\%20PHP\%20is\%20free\%20to\%20download\%20and\%20use}{PHP
  Introduction} (5)
\end{itemize}
\end{frame}

\begin{frame}{Technology}
\protect\hypertarget{technology}{}
\begin{itemize}
\tightlist
\item
  markdown file
\item
  using R Studio to create presentation
\item
  file host at
  \href{https://github.com/jackyhuynh/complete-solutions-for-php-testing-using-simpletest}{Simple
  Test}
\end{itemize}
\end{frame}

\end{document}
